% Options for packages loaded elsewhere
\PassOptionsToPackage{unicode}{hyperref}
\PassOptionsToPackage{hyphens}{url}
%
\documentclass[
  12pt,
]{article}
\usepackage{amsmath,amssymb}
\usepackage{lmodern}
\usepackage{setspace}
\usepackage{iftex}
\ifPDFTeX
  \usepackage[T1]{fontenc}
  \usepackage[utf8]{inputenc}
  \usepackage{textcomp} % provide euro and other symbols
\else % if luatex or xetex
  \usepackage{unicode-math}
  \defaultfontfeatures{Scale=MatchLowercase}
  \defaultfontfeatures[\rmfamily]{Ligatures=TeX,Scale=1}
  \setmainfont[]{Arial}
\fi
% Use upquote if available, for straight quotes in verbatim environments
\IfFileExists{upquote.sty}{\usepackage{upquote}}{}
\IfFileExists{microtype.sty}{% use microtype if available
  \usepackage[]{microtype}
  \UseMicrotypeSet[protrusion]{basicmath} % disable protrusion for tt fonts
}{}
\makeatletter
\@ifundefined{KOMAClassName}{% if non-KOMA class
  \IfFileExists{parskip.sty}{%
    \usepackage{parskip}
  }{% else
    \setlength{\parindent}{0pt}
    \setlength{\parskip}{6pt plus 2pt minus 1pt}}
}{% if KOMA class
  \KOMAoptions{parskip=half}}
\makeatother
\usepackage{xcolor}
\usepackage[margin=1in,margin=1in]{geometry}
\usepackage{color}
\usepackage{fancyvrb}
\newcommand{\VerbBar}{|}
\newcommand{\VERB}{\Verb[commandchars=\\\{\}]}
\DefineVerbatimEnvironment{Highlighting}{Verbatim}{commandchars=\\\{\}}
% Add ',fontsize=\small' for more characters per line
\usepackage{framed}
\definecolor{shadecolor}{RGB}{248,248,248}
\newenvironment{Shaded}{\begin{snugshade}}{\end{snugshade}}
\newcommand{\AlertTok}[1]{\textcolor[rgb]{0.94,0.16,0.16}{#1}}
\newcommand{\AnnotationTok}[1]{\textcolor[rgb]{0.56,0.35,0.01}{\textbf{\textit{#1}}}}
\newcommand{\AttributeTok}[1]{\textcolor[rgb]{0.77,0.63,0.00}{#1}}
\newcommand{\BaseNTok}[1]{\textcolor[rgb]{0.00,0.00,0.81}{#1}}
\newcommand{\BuiltInTok}[1]{#1}
\newcommand{\CharTok}[1]{\textcolor[rgb]{0.31,0.60,0.02}{#1}}
\newcommand{\CommentTok}[1]{\textcolor[rgb]{0.56,0.35,0.01}{\textit{#1}}}
\newcommand{\CommentVarTok}[1]{\textcolor[rgb]{0.56,0.35,0.01}{\textbf{\textit{#1}}}}
\newcommand{\ConstantTok}[1]{\textcolor[rgb]{0.00,0.00,0.00}{#1}}
\newcommand{\ControlFlowTok}[1]{\textcolor[rgb]{0.13,0.29,0.53}{\textbf{#1}}}
\newcommand{\DataTypeTok}[1]{\textcolor[rgb]{0.13,0.29,0.53}{#1}}
\newcommand{\DecValTok}[1]{\textcolor[rgb]{0.00,0.00,0.81}{#1}}
\newcommand{\DocumentationTok}[1]{\textcolor[rgb]{0.56,0.35,0.01}{\textbf{\textit{#1}}}}
\newcommand{\ErrorTok}[1]{\textcolor[rgb]{0.64,0.00,0.00}{\textbf{#1}}}
\newcommand{\ExtensionTok}[1]{#1}
\newcommand{\FloatTok}[1]{\textcolor[rgb]{0.00,0.00,0.81}{#1}}
\newcommand{\FunctionTok}[1]{\textcolor[rgb]{0.00,0.00,0.00}{#1}}
\newcommand{\ImportTok}[1]{#1}
\newcommand{\InformationTok}[1]{\textcolor[rgb]{0.56,0.35,0.01}{\textbf{\textit{#1}}}}
\newcommand{\KeywordTok}[1]{\textcolor[rgb]{0.13,0.29,0.53}{\textbf{#1}}}
\newcommand{\NormalTok}[1]{#1}
\newcommand{\OperatorTok}[1]{\textcolor[rgb]{0.81,0.36,0.00}{\textbf{#1}}}
\newcommand{\OtherTok}[1]{\textcolor[rgb]{0.56,0.35,0.01}{#1}}
\newcommand{\PreprocessorTok}[1]{\textcolor[rgb]{0.56,0.35,0.01}{\textit{#1}}}
\newcommand{\RegionMarkerTok}[1]{#1}
\newcommand{\SpecialCharTok}[1]{\textcolor[rgb]{0.00,0.00,0.00}{#1}}
\newcommand{\SpecialStringTok}[1]{\textcolor[rgb]{0.31,0.60,0.02}{#1}}
\newcommand{\StringTok}[1]{\textcolor[rgb]{0.31,0.60,0.02}{#1}}
\newcommand{\VariableTok}[1]{\textcolor[rgb]{0.00,0.00,0.00}{#1}}
\newcommand{\VerbatimStringTok}[1]{\textcolor[rgb]{0.31,0.60,0.02}{#1}}
\newcommand{\WarningTok}[1]{\textcolor[rgb]{0.56,0.35,0.01}{\textbf{\textit{#1}}}}
\usepackage{graphicx}
\makeatletter
\def\maxwidth{\ifdim\Gin@nat@width>\linewidth\linewidth\else\Gin@nat@width\fi}
\def\maxheight{\ifdim\Gin@nat@height>\textheight\textheight\else\Gin@nat@height\fi}
\makeatother
% Scale images if necessary, so that they will not overflow the page
% margins by default, and it is still possible to overwrite the defaults
% using explicit options in \includegraphics[width, height, ...]{}
\setkeys{Gin}{width=\maxwidth,height=\maxheight,keepaspectratio}
% Set default figure placement to htbp
\makeatletter
\def\fps@figure{htbp}
\makeatother
\setlength{\emergencystretch}{3em} % prevent overfull lines
\providecommand{\tightlist}{%
  \setlength{\itemsep}{0pt}\setlength{\parskip}{0pt}}
\setcounter{secnumdepth}{-\maxdimen} % remove section numbering
\ifLuaTeX
  \usepackage{selnolig}  % disable illegal ligatures
\fi
\IfFileExists{bookmark.sty}{\usepackage{bookmark}}{\usepackage{hyperref}}
\IfFileExists{xurl.sty}{\usepackage{xurl}}{} % add URL line breaks if available
\urlstyle{same} % disable monospaced font for URLs
\hypersetup{
  hidelinks,
  pdfcreator={LaTeX via pandoc}}

\title{Universidad de San Andrés\\
Departamento de Matemática y Ciencias\\
Maestría en Ciencia de Datos}
\usepackage{etoolbox}
\makeatletter
\providecommand{\subtitle}[1]{% add subtitle to \maketitle
  \apptocmd{\@title}{\par {\large #1 \par}}{}{}
}
\makeatother
\subtitle{TP Final\\
Asignatura: Estadística Espacial}
\author{true \and true \and true}
\date{14-04-2023}

\begin{document}
\maketitle

{
\setcounter{tocdepth}{2}
\tableofcontents
}
\setstretch{1.5}
\hypertarget{problemuxe1tica}{%
\section{Problemática}\label{problemuxe1tica}}

Para este trabajo, contamos con la base de datos ``meuse'' de la
librería ``sp'' de R como objeto de análisis. En particular, nos
encontramos con datos sobre las locaciones de grandes concentraciones de
metales pesados, más precisamente zinc, en la capa superficial del suelo
de la llanura del río Meuse. Es por eso que, valiéndonos de los recursos
vistos en el curso, decidimos plantear un análisis sistemático que
abarque desde una estadística descriptiva, seguido por estudios de
estacionariedad, isotropía, análisis estructural y predicción.

\hypertarget{libreruxedas}{%
\section{Librerías}\label{libreruxedas}}

\begin{Shaded}
\begin{Highlighting}[]
\FunctionTok{library}\NormalTok{(}\StringTok{\textquotesingle{}geoR\textquotesingle{}}\NormalTok{)}
\FunctionTok{library}\NormalTok{(}\StringTok{\textquotesingle{}spdep\textquotesingle{}}\NormalTok{)}
\FunctionTok{library}\NormalTok{(}\StringTok{\textquotesingle{}gstat\textquotesingle{}}\NormalTok{)}
\FunctionTok{library}\NormalTok{(}\StringTok{"leafsync"}\NormalTok{)}
\FunctionTok{library}\NormalTok{(}\StringTok{\textquotesingle{}mapview\textquotesingle{}}\NormalTok{)}
\FunctionTok{library}\NormalTok{(}\StringTok{\textquotesingle{}leaflet\textquotesingle{}}\NormalTok{)}
\FunctionTok{library}\NormalTok{(}\StringTok{\textquotesingle{}RColorBrewer\textquotesingle{}}\NormalTok{)}
\FunctionTok{library}\NormalTok{(}\StringTok{\textquotesingle{}ggplot2\textquotesingle{}}\NormalTok{)}
\FunctionTok{library}\NormalTok{(}\StringTok{\textquotesingle{}ggmap\textquotesingle{}}\NormalTok{)}
\FunctionTok{library}\NormalTok{(}\StringTok{\textquotesingle{}tibble\textquotesingle{}}\NormalTok{)}
\FunctionTok{library}\NormalTok{(}\StringTok{\textquotesingle{}caret\textquotesingle{}}\NormalTok{)}
\FunctionTok{library}\NormalTok{(}\StringTok{\textquotesingle{}sf\textquotesingle{}}\NormalTok{)}
\FunctionTok{library}\NormalTok{(}\StringTok{\textquotesingle{}sp\textquotesingle{}}\NormalTok{)}
\FunctionTok{library}\NormalTok{(}\StringTok{\textquotesingle{}PerformanceAnalytics\textquotesingle{}}\NormalTok{)}
\FunctionTok{library}\NormalTok{(}\StringTok{"rgdal"}\NormalTok{)}
\FunctionTok{library}\NormalTok{(}\StringTok{\textquotesingle{}lattice\textquotesingle{}}\NormalTok{) }
\FunctionTok{library}\NormalTok{(}\StringTok{\textquotesingle{}grDevices\textquotesingle{}}\NormalTok{)}
\FunctionTok{library}\NormalTok{(}\StringTok{\textquotesingle{}GGally\textquotesingle{}}\NormalTok{)}
\FunctionTok{library}\NormalTok{(}\StringTok{\textquotesingle{}dplyr\textquotesingle{}}\NormalTok{)}
\FunctionTok{library}\NormalTok{(}\StringTok{\textquotesingle{}geostan\textquotesingle{}}\NormalTok{)}
\end{Highlighting}
\end{Shaded}

\hypertarget{metodologuxeda}{%
\section{Metodología}\label{metodologuxeda}}

En un primer lugar, realizamos la carga de los datos y generamos una
copia de seguridad de la misma en caso de que necesitemos modificarla
para algún análisis particular.

\begin{Shaded}
\begin{Highlighting}[]
\FunctionTok{data}\NormalTok{(meuse)}
\NormalTok{copia\_seguridad }\OtherTok{\textless{}{-}}\NormalTok{ meuse}
\end{Highlighting}
\end{Shaded}

Comenzamos el análisis exploratorio de la base en cuestión. Vemos que
consta de 155 registros, y que ninguno cuenta con valores nulos o
faltantes.

\begin{Shaded}
\begin{Highlighting}[]
\CommentTok{\# Vemos la cantidad de registros y columnas}
\FunctionTok{class}\NormalTok{(meuse[,}\FunctionTok{c}\NormalTok{(}\StringTok{\textquotesingle{}x\textquotesingle{}}\NormalTok{,}\StringTok{\textquotesingle{}y\textquotesingle{}}\NormalTok{,}\StringTok{\textquotesingle{}zinc\textquotesingle{}}\NormalTok{)]) }\CommentTok{\# Es de clase "data frame"}
\end{Highlighting}
\end{Shaded}

\begin{verbatim}
## [1] "data.frame"
\end{verbatim}

\begin{Shaded}
\begin{Highlighting}[]
\FunctionTok{dim}\NormalTok{(meuse[,}\FunctionTok{c}\NormalTok{(}\StringTok{\textquotesingle{}x\textquotesingle{}}\NormalTok{,}\StringTok{\textquotesingle{}y\textquotesingle{}}\NormalTok{,}\StringTok{\textquotesingle{}zinc\textquotesingle{}}\NormalTok{)]) }
\end{Highlighting}
\end{Shaded}

\begin{verbatim}
## [1] 155   3
\end{verbatim}

\begin{Shaded}
\begin{Highlighting}[]
\CommentTok{\# Analizamos duplicados y valores nulos}
\FunctionTok{sum}\NormalTok{(}\FunctionTok{duplicated}\NormalTok{(meuse[,}\FunctionTok{c}\NormalTok{(}\StringTok{\textquotesingle{}x\textquotesingle{}}\NormalTok{,}\StringTok{\textquotesingle{}y\textquotesingle{}}\NormalTok{,}\StringTok{\textquotesingle{}zinc\textquotesingle{}}\NormalTok{)])) }\CommentTok{\# No tenemos registros duplicados}
\end{Highlighting}
\end{Shaded}

\begin{verbatim}
## [1] 0
\end{verbatim}

\begin{Shaded}
\begin{Highlighting}[]
\FunctionTok{sum}\NormalTok{(}\FunctionTok{is.na}\NormalTok{(meuse[,}\FunctionTok{c}\NormalTok{(}\StringTok{\textquotesingle{}x\textquotesingle{}}\NormalTok{,}\StringTok{\textquotesingle{}y\textquotesingle{}}\NormalTok{,}\StringTok{\textquotesingle{}zinc\textquotesingle{}}\NormalTok{)])) }\CommentTok{\# No tenemos valores NA. }
\end{Highlighting}
\end{Shaded}

\begin{verbatim}
## [1] 0
\end{verbatim}

\begin{Shaded}
\begin{Highlighting}[]
\CommentTok{\# Vemos algunos datos de la base}
\FunctionTok{head}\NormalTok{(meuse[,}\FunctionTok{c}\NormalTok{(}\StringTok{\textquotesingle{}x\textquotesingle{}}\NormalTok{,}\StringTok{\textquotesingle{}y\textquotesingle{}}\NormalTok{,}\StringTok{\textquotesingle{}zinc\textquotesingle{}}\NormalTok{)])}
\end{Highlighting}
\end{Shaded}

\begin{verbatim}
##        x      y zinc
## 1 181072 333611 1022
## 2 181025 333558 1141
## 3 181165 333537  640
## 4 181298 333484  257
## 5 181307 333330  269
## 6 181390 333260  281
\end{verbatim}

Para entender más la base, hicimos una estadística descriptiva
encontrando que existen valores atípicos de zinc, ya que se alcanzan
niveles de concentración de 1839 ppm en una muestra donde el tercer
cuartil se alcanza a las 674.5 ppm.

\begin{Shaded}
\begin{Highlighting}[]
\CommentTok{\# Vemos la estadística descriptiva}
\FunctionTok{summary}\NormalTok{(meuse[,}\FunctionTok{c}\NormalTok{(}\StringTok{\textquotesingle{}x\textquotesingle{}}\NormalTok{,}\StringTok{\textquotesingle{}y\textquotesingle{}}\NormalTok{,}\StringTok{\textquotesingle{}zinc\textquotesingle{}}\NormalTok{)]) }\CommentTok{\# Tenemos valores máximos que superan ampliamente la mediana de la población. }
\end{Highlighting}
\end{Shaded}

\begin{verbatim}
##        x                y               zinc       
##  Min.   :178605   Min.   :329714   Min.   : 113.0  
##  1st Qu.:179371   1st Qu.:330762   1st Qu.: 198.0  
##  Median :179991   Median :331633   Median : 326.0  
##  Mean   :180005   Mean   :331635   Mean   : 469.7  
##  3rd Qu.:180630   3rd Qu.:332463   3rd Qu.: 674.5  
##  Max.   :181390   Max.   :333611   Max.   :1839.0
\end{verbatim}

\begin{Shaded}
\begin{Highlighting}[]
\CommentTok{\# Sospechamos entonces que habrán eventuales datos atípicos}
\end{Highlighting}
\end{Shaded}

Además, estudiamos la distribución de la concentración del metal y su
relación con las coordenadas registradas en la siguiente salida.

\begin{Shaded}
\begin{Highlighting}[]
\NormalTok{x}\OtherTok{\textless{}{-}}\NormalTok{meuse}\SpecialCharTok{$}\NormalTok{x}
\NormalTok{y}\OtherTok{\textless{}{-}}\NormalTok{meuse}\SpecialCharTok{$}\NormalTok{y}
\NormalTok{zinc}\OtherTok{\textless{}{-}}\NormalTok{meuse}\SpecialCharTok{$}\NormalTok{zinc}
\NormalTok{data }\OtherTok{\textless{}{-}} \FunctionTok{cbind}\NormalTok{(x,y,zinc)}
\NormalTok{datag }\OtherTok{\textless{}{-}} \FunctionTok{as.geodata}\NormalTok{(data) }\CommentTok{\# Lo llevamos a tipo "geo{-}data"}
\FunctionTok{plot}\NormalTok{(datag)}
\end{Highlighting}
\end{Shaded}

\includegraphics{trabajo_final_files/figure-latex/unnamed-chunk-7-1.pdf}

\begin{Shaded}
\begin{Highlighting}[]
\CommentTok{\#data(meuse)}
\CommentTok{\#coordinates(meuse) \textless{}{-} \textasciitilde{}x+y}
\CommentTok{\#proj4string(meuse) \textless{}{-} CRS("+init=epsg:28992")}

\CommentTok{\#mapview(meuse, zcol = c("zinc"), legend = TRUE)}
\end{Highlighting}
\end{Shaded}

Aquí podemos observar que todos los puntos se encuentran dispersos a lo
largo del río, teniendo ciertos lugares donde los niveles de
concentración del metal parecen agruparse. Por ejemplo, para el caso de
los puntos rojos, notamos que estos son los valores que reflejan el
mayor nivel de concentración de zinc, y que están agrupados muy próximos
al río; mientras que los puntos que con menores niveles de concentración
se hallan más dispersos y a una distancia mayor en relación al río.

Haciendo referencia a la relación entre la variable con cada coordenada,
podemos notar una muy leve correlación entre los niveles de
concentración por zinc con las coordenada x e y, tratándose en ambos
casos de una magnitud muy chica y sin significancia estadística para los
niveles usualmente utilizados.

\begin{Shaded}
\begin{Highlighting}[]
\FunctionTok{data}\NormalTok{(meuse)}
\CommentTok{\# Definimos una función para ver las correlaciones en conjunto con los plots}
\NormalTok{my\_fn }\OtherTok{\textless{}{-}} \ControlFlowTok{function}\NormalTok{(data, mapping, ...)\{}
\NormalTok{  p }\OtherTok{\textless{}{-}} \FunctionTok{ggplot}\NormalTok{(}\AttributeTok{data =}\NormalTok{ data, }\AttributeTok{mapping =}\NormalTok{ mapping) }\SpecialCharTok{+} 
    \FunctionTok{geom\_point}\NormalTok{() }\SpecialCharTok{+} 
    \FunctionTok{geom\_smooth}\NormalTok{(}\AttributeTok{method=}\NormalTok{loess, }\AttributeTok{fill=}\StringTok{"red"}\NormalTok{, }\AttributeTok{color=}\StringTok{"red"}\NormalTok{, ...) }\SpecialCharTok{+}
    \FunctionTok{geom\_smooth}\NormalTok{(}\AttributeTok{method=}\NormalTok{lm, }\AttributeTok{fill=}\StringTok{"blue"}\NormalTok{, }\AttributeTok{color=}\StringTok{"blue"}\NormalTok{, ...)}
\NormalTok{  p}
\NormalTok{\}}

\NormalTok{g }\OtherTok{=} \FunctionTok{ggpairs}\NormalTok{(meuse[,}\FunctionTok{c}\NormalTok{(}\StringTok{\textquotesingle{}x\textquotesingle{}}\NormalTok{,}\StringTok{\textquotesingle{}y\textquotesingle{}}\NormalTok{,}\StringTok{\textquotesingle{}zinc\textquotesingle{}}\NormalTok{)],}\AttributeTok{columns =} \DecValTok{1}\SpecialCharTok{:}\DecValTok{3}\NormalTok{, }\AttributeTok{lower =} \FunctionTok{list}\NormalTok{(}\AttributeTok{continuous =}\NormalTok{ my\_fn))}
\NormalTok{g}
\end{Highlighting}
\end{Shaded}

\includegraphics{trabajo_final_files/figure-latex/unnamed-chunk-9-1.pdf}

Respecto a la distribución del zinc, para alcanzar una mayor
profundidad, hicimos el siguiente test de Shapiro-Wilk para evaluar su
normalidad de la distribución con mayor rigurosidad.

\[H_0 = Datos \thicksim N(\mu,\sigma) ~~~~ vs ~~~~ H_1 = Datos \nsim N(\mu,\sigma) \]

Como resultado, obtuvimos un p-valor de 3.28e-12, por lo cual rechazamos
la hipótesis nula de normalidad. También hicimos el ejercicio
transformando la variable con logaritmo, y obtuvimos un resultado muy
similar a favor de la no normalidad de los datos. Esto también se
evidencia si comparamos los cuantiles de una normal con la distribución
de los datos en un QQ-Plot. Luego, como tanto los test de Moran como el
de Geary presuponen una distribución normal de los datos, tendremos que
hacer ambos tests considerando sus versiones aleatorizadas.

\begin{Shaded}
\begin{Highlighting}[]
\FunctionTok{shapiro.test}\NormalTok{(meuse}\SpecialCharTok{$}\NormalTok{zinc)}
\end{Highlighting}
\end{Shaded}

\begin{verbatim}
## 
##  Shapiro-Wilk normality test
## 
## data:  meuse$zinc
## W = 0.82812, p-value = 3.28e-12
\end{verbatim}

\begin{Shaded}
\begin{Highlighting}[]
\CommentTok{\# Vemos como parece quedar con una distribución más cercanaa a la normal si lo llevamos a logaritmo}
\NormalTok{meuse}\SpecialCharTok{$}\NormalTok{lnZn }\OtherTok{\textless{}{-}} \FunctionTok{log}\NormalTok{(meuse}\SpecialCharTok{$}\NormalTok{zinc, }\AttributeTok{base =} \FunctionTok{exp}\NormalTok{(}\DecValTok{1}\NormalTok{))}
\FunctionTok{shapiro.test}\NormalTok{(meuse}\SpecialCharTok{$}\NormalTok{lnZn)}
\end{Highlighting}
\end{Shaded}

\begin{verbatim}
## 
##  Shapiro-Wilk normality test
## 
## data:  meuse$lnZn
## W = 0.95092, p-value = 2.901e-05
\end{verbatim}

\begin{Shaded}
\begin{Highlighting}[]
\CommentTok{\# Igual rechazamos normalidad}
\end{Highlighting}
\end{Shaded}

\begin{Shaded}
\begin{Highlighting}[]
\NormalTok{df }\OtherTok{=} \FunctionTok{as.data.frame}\NormalTok{(zinc)}
\FunctionTok{ggplot}\NormalTok{(df, }\FunctionTok{aes}\NormalTok{(}\AttributeTok{sample=}\NormalTok{zinc)) }\SpecialCharTok{+} \FunctionTok{stat\_qq}\NormalTok{() }\SpecialCharTok{+} \FunctionTok{stat\_qq\_line}\NormalTok{() }\SpecialCharTok{+} \FunctionTok{labs}\NormalTok{(}\AttributeTok{title =} \StringTok{\textquotesingle{}QQ Plot de la distribución del Zinc\textquotesingle{}}\NormalTok{) }\SpecialCharTok{+} \FunctionTok{theme}\NormalTok{(}\AttributeTok{plot.title =} \FunctionTok{element\_text}\NormalTok{(}\AttributeTok{hjust=}\FloatTok{0.5}\NormalTok{))}
\end{Highlighting}
\end{Shaded}

\includegraphics{trabajo_final_files/figure-latex/unnamed-chunk-12-1.pdf}

\hypertarget{moran---geary}{%
\subsection{Moran - Geary}\label{moran---geary}}

\hypertarget{moral-analuxedsis-global-y-local}{%
\subsubsection{Moral: Analísis global y
local}\label{moral-analuxedsis-global-y-local}}

Para continuar con el análisis geoestadístico, apelamos a los índices y
tests vistos en clase, analizando así si es que existe -o no- una
autocorrelación espacial.

En la primera parte, para poder calcular ambos índices para determinar
la autocorrelación de nuestros datos,debemos definir los pesos para cada
una de las observaciones para así definir una grilla la cual nos
determine el vecindario y los vecinos de cada punto.

\begin{Shaded}
\begin{Highlighting}[]
\FunctionTok{data}\NormalTok{(meuse)}
\FunctionTok{coordinates}\NormalTok{(meuse) }\OtherTok{\textless{}{-}} \FunctionTok{c}\NormalTok{(}\StringTok{"x"}\NormalTok{, }\StringTok{"y"}\NormalTok{)}
\NormalTok{coordenadas }\OtherTok{\textless{}{-}} \FunctionTok{coordinates}\NormalTok{(meuse)}
\NormalTok{grilla }\OtherTok{\textless{}{-}} \FunctionTok{dnearneigh}\NormalTok{(meuse,}\DecValTok{0}\NormalTok{,}\DecValTok{400}\NormalTok{)}
\NormalTok{pesos }\OtherTok{\textless{}{-}} \FunctionTok{nb2listw}\NormalTok{(grilla, }\AttributeTok{style =} \StringTok{"W"}\NormalTok{)}
\FunctionTok{plot}\NormalTok{(grilla, coordenadas, }\AttributeTok{col =} \StringTok{"red"}\NormalTok{, }\AttributeTok{pch =} \DecValTok{19}\NormalTok{, }\AttributeTok{cex =} \DecValTok{1}\NormalTok{)}
\end{Highlighting}
\end{Shaded}

\includegraphics{trabajo_final_files/figure-latex/unnamed-chunk-13-1.pdf}

Para evaluar la autocorrelación espacial en mayor profundidad, vamos a
calcular los índices local y global de Moran como también el de Geary.
Así buscaremos comprender la variación del fenómeno, en este caso del
zinc en el río Meuse, en un marco geográfico de análisis. Los posibles
resultados son:

\begin{itemize}
\item
  Notar que hay evidencia a favor de la existencia de autocorrelación
  positiva. En este caso deberíamos notar que el zinc se agrupa en zonas
  uniformes conformando de esta manera una especie de cluster.
\item
  Hallar evidencia de autocorrelación negativa, notando que el zinc se
  encuentra disperso, es decir que la presencia de zinc sea disímil en
  sus lugares aledaños/vecinos.
\item
  No encontrar evidencia de autocorrelación espacial, lo que nos diría
  que la variable tiene un comportamiento aleatorio en el estudio del
  fenómeno en el terreno.
\end{itemize}

En una primera instancia calculamos el Índice de Moran Global mediante
la ejecución del test de Moran Global en su versión aleatorizada. Estos,
independientemente de la opción que elijamos para calcular los pesos,
nos arroja resultados que nos permiten llegar a las mismas conclusiones:

\begin{itemize}
\item
  Obtenemos un p-valor menor a los niveles estadísticos usualmente
  utilizados, permitiendo concluir que hay evidencia suficiente para
  rechazar la hipótesis nula de que no existe autocorrelación espacial.
\item
  Se obtiene un valor de índice de Moran (IM) superior a su valor
  esperado, lo cual reafirma la conclusión de que existe autocorrelación
  espacial.
\end{itemize}

\begin{Shaded}
\begin{Highlighting}[]
\FunctionTok{moran.test}\NormalTok{(meuse}\SpecialCharTok{$}\NormalTok{zinc, }\FunctionTok{nb2listw}\NormalTok{(grilla, }\AttributeTok{style =} \StringTok{"W"}\NormalTok{), }\AttributeTok{randomisation =} \ConstantTok{TRUE}\NormalTok{)}
\end{Highlighting}
\end{Shaded}

\begin{verbatim}
## 
##  Moran I test under randomisation
## 
## data:  meuse$zinc  
## weights: nb2listw(grilla, style = "W")    
## 
## Moran I statistic standard deviate = 10.188, p-value < 2.2e-16
## alternative hypothesis: greater
## sample estimates:
## Moran I statistic       Expectation          Variance 
##       0.320600991      -0.006493506       0.001030686
\end{verbatim}

\begin{Shaded}
\begin{Highlighting}[]
\CommentTok{\#moran.test(meuse$zinc, nb2listw(grilla, style = "S"), randomisation = TRUE)}
\CommentTok{\#moran.test(meuse$zinc, nb2listw(grilla, style = "B"), randomisation = TRUE)}
\CommentTok{\#moran.test(meuse$zinc, nb2listw(grilla, style = "C"), randomisation = TRUE)}
\CommentTok{\#moran.test(meuse$zinc, nb2listw(grilla, style = "U"), randomisation = TRUE)}
\CommentTok{\#moran.test(meuse$zinc, nb2listw(grilla, style = "minmax"), randomisation = TRUE)}
\end{Highlighting}
\end{Shaded}

De manera grafica, también podemos evaluar cuan similar es cada dato con
respecto a sus vecinos. A través de este índice podemos encontrar datos
atípicos, es decir posibles outliers espaciales.

\begin{Shaded}
\begin{Highlighting}[]
\NormalTok{M }\OtherTok{\textless{}{-}} \FunctionTok{moran.plot}\NormalTok{(meuse}\SpecialCharTok{$}\NormalTok{zinc,pesos,}\AttributeTok{zero.policy=}\NormalTok{F,}\AttributeTok{col=}\DecValTok{3}\NormalTok{, }\AttributeTok{quiet=}\ConstantTok{TRUE}\NormalTok{,}\AttributeTok{labels=}\NormalTok{T,}\AttributeTok{xlab =} \StringTok{"zinc"}\NormalTok{, }\AttributeTok{ylab=}\StringTok{"lag(zinc)"}\NormalTok{)}
\end{Highlighting}
\end{Shaded}

\includegraphics{trabajo_final_files/figure-latex/unnamed-chunk-15-1.pdf}

Encontramos luego que hay eventuales candidatos a outliers espaciales,
como son por ejemplo los registros 69 y 118, los cuales muestran un
valor muy distinto a sus observaciones cercanas. Además, si vemos las
observaciones que tienen un Índice de Moran Local (IML) muy bajo a nivel
observación, nos encontramos con las observaciones que sospechamos
outliers de manera gráfica muestran efectivamente un IML bajo:

\begin{Shaded}
\begin{Highlighting}[]
\NormalTok{ML }\OtherTok{\textless{}{-}} \FunctionTok{localmoran}\NormalTok{(meuse}\SpecialCharTok{$}\NormalTok{zinc, pesos, }\AttributeTok{alternative =}\StringTok{"less"}\NormalTok{)}
\NormalTok{IML }\OtherTok{\textless{}{-}}\FunctionTok{data.frame}\NormalTok{(ML,}\AttributeTok{row.names=}\NormalTok{elevation}\SpecialCharTok{$}\NormalTok{Casos)}

\CommentTok{\#Potenciales outliers}
\NormalTok{potenciales\_outliers }\OtherTok{=} \FunctionTok{subset}\NormalTok{(IML, IML}\SpecialCharTok{$}\NormalTok{Ii }\SpecialCharTok{\textless{}} \DecValTok{0} \SpecialCharTok{\&}\NormalTok{ IML}\SpecialCharTok{$}\NormalTok{Pr.z...E.Ii.. }\SpecialCharTok{\textless{}} \FloatTok{0.05}\NormalTok{)}
\NormalTok{potenciales\_outliers}
\end{Highlighting}
\end{Shaded}

\begin{verbatim}
##              Ii          E.Ii       Var.Ii      Z.Ii Pr.z...E.Ii..
## 41  -0.22547290 -0.0014302294 0.0109317773 -2.142818  0.0160638455
## 43  -0.32244456 -0.0041844569 0.0299942797 -1.837651  0.0330569422
## 57  -0.07890771 -0.0000689982 0.0006476965 -3.097803  0.0009748063
## 69  -0.66536542 -0.0046098322 0.0748935167 -2.414453  0.0078794310
## 98  -0.38567470 -0.0039043753 0.0263977957 -2.349732  0.0093934649
## 121 -0.51601760 -0.0053051084 0.0495390824 -2.294576  0.0108787093
## 122 -0.41767062 -0.0022780643 0.0299336116 -2.400928  0.0081767750
## 125 -0.31610960 -0.0020927818 0.0141751997 -2.637475  0.0041762915
\end{verbatim}

\begin{Shaded}
\begin{Highlighting}[]
\CommentTok{\# Nos quedamos con la base sin outliers}
\NormalTok{base\_sin\_outliers }\OtherTok{=}\NormalTok{ copia\_seguridad[}\SpecialCharTok{!}\NormalTok{(}\FunctionTok{rownames}\NormalTok{(copia\_seguridad) }\SpecialCharTok{\%in\%} \FunctionTok{rownames}\NormalTok{(potenciales\_outliers)),]}

\FunctionTok{coordinates}\NormalTok{(base\_sin\_outliers) }\OtherTok{\textless{}{-}} \FunctionTok{c}\NormalTok{(}\StringTok{"x"}\NormalTok{, }\StringTok{"y"}\NormalTok{)}
\NormalTok{coordenadas\_sin\_outliers }\OtherTok{\textless{}{-}} \FunctionTok{coordinates}\NormalTok{(base\_sin\_outliers)}
\NormalTok{grilla\_sin\_outliers }\OtherTok{\textless{}{-}} \FunctionTok{dnearneigh}\NormalTok{(base\_sin\_outliers,}\DecValTok{0}\NormalTok{,}\DecValTok{400}\NormalTok{)}
\NormalTok{pesos\_sin\_outliers }\OtherTok{\textless{}{-}} \FunctionTok{nb2listw}\NormalTok{(grilla\_sin\_outliers, }\AttributeTok{style =} \StringTok{"W"}\NormalTok{)}
\end{Highlighting}
\end{Shaded}

Si volvemos a hacer el gráfico de Moran, de hecho nos cambia la
pendiente, puesto que es un método que es sensible a outliers.

\begin{Shaded}
\begin{Highlighting}[]
\NormalTok{M\_sin\_outliers }\OtherTok{\textless{}{-}} \FunctionTok{moran.plot}\NormalTok{(base\_sin\_outliers}\SpecialCharTok{$}\NormalTok{zinc,pesos\_sin\_outliers,}\AttributeTok{zero.policy=}\NormalTok{F,}\AttributeTok{col=}\DecValTok{3}\NormalTok{, }\AttributeTok{quiet=}\ConstantTok{TRUE}\NormalTok{,}\AttributeTok{labels=}\NormalTok{T,}\AttributeTok{xlab =} \StringTok{"zinc"}\NormalTok{, }\AttributeTok{ylab=}\StringTok{"lag(zinc)"}\NormalTok{)}
\end{Highlighting}
\end{Shaded}

\includegraphics{trabajo_final_files/figure-latex/unnamed-chunk-18-1.pdf}

\begin{Shaded}
\begin{Highlighting}[]
\FunctionTok{moran.test}\NormalTok{(meuse}\SpecialCharTok{$}\NormalTok{zinc, }\FunctionTok{nb2listw}\NormalTok{(grilla, }\AttributeTok{style =} \StringTok{"W"}\NormalTok{), }\AttributeTok{randomisation =} \ConstantTok{TRUE}\NormalTok{) }\CommentTok{\# Test con outliers}
\end{Highlighting}
\end{Shaded}

\begin{verbatim}
## 
##  Moran I test under randomisation
## 
## data:  meuse$zinc  
## weights: nb2listw(grilla, style = "W")    
## 
## Moran I statistic standard deviate = 10.188, p-value < 2.2e-16
## alternative hypothesis: greater
## sample estimates:
## Moran I statistic       Expectation          Variance 
##       0.320600991      -0.006493506       0.001030686
\end{verbatim}

\begin{Shaded}
\begin{Highlighting}[]
\FunctionTok{moran.test}\NormalTok{(base\_sin\_outliers}\SpecialCharTok{$}\NormalTok{zinc, }\FunctionTok{nb2listw}\NormalTok{(grilla\_sin\_outliers, }\AttributeTok{style =} \StringTok{"W"}\NormalTok{), }\AttributeTok{randomisation =} \ConstantTok{TRUE}\NormalTok{) }\CommentTok{\# Test sin outliers}
\end{Highlighting}
\end{Shaded}

\begin{verbatim}
## 
##  Moran I test under randomisation
## 
## data:  base_sin_outliers$zinc  
## weights: nb2listw(grilla_sin_outliers, style = "W")    
## 
## Moran I statistic standard deviate = 9.7328, p-value < 2.2e-16
## alternative hypothesis: greater
## sample estimates:
## Moran I statistic       Expectation          Variance 
##       0.322545398      -0.006849315       0.001145406
\end{verbatim}

\hypertarget{geary}{%
\subsubsection{Geary}\label{geary}}

Con este índice vamos a medir la asociación espacial a través del cambio
y de su varianza. De esta manera evaluaremos si la distribución del zinc
en el río Meuse es aleatoria o no, afirmando que hay una alta
autocorrelación espacial si la diferencia de los valores del zinc en
diferentes lugares son similares y la distancia entre los lugares es
pequeña.

Este índice es más sensible a outliers que el índice de Moran. Toma
valores que van entre 0 y 2, donde:

\begin{itemize}
\item
  1 indica una distribución espacial aleatoria de la variable.
\item
  valores menores que 1 indican una autocorrelación positiva, siendo los
  valores similares cercanos unos a otros.
\item
  valores mayores que 1 indican una autocorrelación negativa, siendo los
  valores diferentes cerca unos de otros.
\end{itemize}

En esta segunda instancia calculamos el Índice de Geary mediante la
ejecución del test de Geary en su versión aleatorizada. Estos,
independientemente de la opción que elijamos para calcular los pesos,
nos arroja resultados que nos permiten llegar a conclusiones similares:

\begin{itemize}
\item
  Obtenemos un p-valor menor a los niveles estadísticos usualmente
  utilizados, permitiendo concluir que hay evidencia suficiente para
  rechazar la hipótesis nula de que no existe autocorrelación espacial.
\item
  Se obtiene un valor de índice de Geary (IG) superior a su valor
  esperado, lo cual reafirma la conclusión de que existe autocorrelación
  espacial.
\end{itemize}

\begin{Shaded}
\begin{Highlighting}[]
\FunctionTok{geary.test}\NormalTok{(meuse}\SpecialCharTok{$}\NormalTok{zinc, }\FunctionTok{nb2listw}\NormalTok{(grilla, }\AttributeTok{style =} \StringTok{"W"}\NormalTok{),}\AttributeTok{randomisation =} \ConstantTok{TRUE}\NormalTok{)}
\end{Highlighting}
\end{Shaded}

\begin{verbatim}
## 
##  Geary C test under randomisation
## 
## data:  meuse$zinc 
## weights: nb2listw(grilla, style = "W") 
## 
## Geary C statistic standard deviate = 9.9183, p-value < 2.2e-16
## alternative hypothesis: Expectation greater than statistic
## sample estimates:
## Geary C statistic       Expectation          Variance 
##       0.633355716       1.000000000       0.001366521
\end{verbatim}

\begin{Shaded}
\begin{Highlighting}[]
\CommentTok{\#geary.test(meuse$zinc, nb2listw(grilla, style = "S"),randomisation = TRUE)}
\CommentTok{\#geary.test(meuse$zinc, nb2listw(grilla, style = "B"),randomisation = TRUE)}
\CommentTok{\#geary.test(meuse$zinc, nb2listw(grilla, style = "C"),randomisation = TRUE)}
\CommentTok{\#geary.test(meuse$zinc, nb2listw(grilla, style = "U"),randomisation = TRUE)}
\CommentTok{\#geary.test(meuse$zinc, nb2listw(grilla, style = "minmax"),randomisation = TRUE) }
\end{Highlighting}
\end{Shaded}

En base a los resultados obtenidos nos indica que en sitios conectados
los valores del zinc son similares. Además, obtenemos los mismos
resultados en caso de purgar los valores que sospechamos outliers
previamente.

\begin{Shaded}
\begin{Highlighting}[]
\FunctionTok{geary.test}\NormalTok{(base\_sin\_outliers}\SpecialCharTok{$}\NormalTok{zinc, }\FunctionTok{nb2listw}\NormalTok{(grilla\_sin\_outliers, }\AttributeTok{style =} \StringTok{"W"}\NormalTok{), }\AttributeTok{randomisation =} \ConstantTok{TRUE}\NormalTok{)}
\end{Highlighting}
\end{Shaded}

\begin{verbatim}
## 
##  Geary C test under randomisation
## 
## data:  base_sin_outliers$zinc 
## weights: nb2listw(grilla_sin_outliers, style = "W") 
## 
## Geary C statistic standard deviate = 9.4415, p-value < 2.2e-16
## alternative hypothesis: Expectation greater than statistic
## sample estimates:
## Geary C statistic       Expectation          Variance 
##       0.626140212       1.000000000       0.001567976
\end{verbatim}

Al arrojar valores menores a 1, podemos afirmar que hay autocorrelación
positiva, es decir que los valores similares tienden a estar cerca unos
de otros.

\hypertarget{variograma}{%
\subsection{Variograma}\label{variograma}}

El variograma es una herramienta para el análisis geoestadístico
utilizada para describir la variabilidad espacial de un fenómeno, en
este caso el zinc. Hay tres tipos de variogramas que se utilizaran para
el análisis:

\begin{itemize}
\item
  \textbf{Variograma nube:} podemos graficar la distancia de los puntos
  en el espacio versus la variable de cambio (semi varianza al
  cuadrado).
\item
  \textbf{Variograma empírico o experimental:} Como con el variograma
  nube no alcanza vamos a calcular el variograma empírico, el cual va a
  ser construido a partir del variograma nube, dividiendo el eje de
  distancias en intervalos, tomando un representante sobre el cual
  promedió con todos los puntos que caen dentro del intervalo,
  consiguiendo así la representación del mismo.
\item
  \textbf{Variograma teórico:} este será calculado dado que necesitamos
  de una línea continúa para poder estimar las observaciones que no
  fueron representadas.
\end{itemize}

Tras haber realizado tanto los cálculos de los índices como los análisis
de autocorrelación y la evaluación de potenciales outliers, procedemos
con el \textbf{análisis estructural}. Para esto construiremos el
\textbf{variograma nube}, tomando el dataset sin las observaciones que
consideramos como outliers.

A continuación mostraremos cómo es la visualización tanto del variograma
nube como del empírico.

\begin{Shaded}
\begin{Highlighting}[]
\FunctionTok{par}\NormalTok{(}\AttributeTok{mfrow=}\FunctionTok{c}\NormalTok{(}\DecValTok{1}\NormalTok{,}\DecValTok{2}\NormalTok{))}
\FunctionTok{plot}\NormalTok{(nube\_clasica, }\AttributeTok{col =} \StringTok{\textquotesingle{}cadetblue\textquotesingle{}}\NormalTok{, }\AttributeTok{main =} \StringTok{"Variograma Nube}\SpecialCharTok{\textbackslash{}n}\StringTok{ {-} classical estimator"}\NormalTok{)}
\FunctionTok{plot}\NormalTok{(nube\_CH,}\AttributeTok{col =} \StringTok{\textquotesingle{}cadetblue\textquotesingle{}}\NormalTok{,  }\AttributeTok{main =} \StringTok{"Variograma Nube}\SpecialCharTok{\textbackslash{}n}\StringTok{ {-}modulus estimator"}\NormalTok{)}
\end{Highlighting}
\end{Shaded}

\includegraphics{trabajo_final_files/figure-latex/unnamed-chunk-23-1.pdf}

\begin{Shaded}
\begin{Highlighting}[]
\FunctionTok{par}\NormalTok{(}\AttributeTok{mfrow =} \FunctionTok{c}\NormalTok{(}\DecValTok{1}\NormalTok{,}\DecValTok{1}\NormalTok{))}
\end{Highlighting}
\end{Shaded}

Lo primero que podemos observar es que no parece haber una tendencia
clara. Esto era de esperarse luego de lo observado en el análisis
exploratorio previo, donde las correlaciones entre las coordenadas y los
niveles de concentración de zinc eran débiles.

Luego, para analizar la isotropía y saber si es correcto que nuestro
variograma sea omnidireccional, realizamos el variograma mapa.

\begin{Shaded}
\begin{Highlighting}[]
\NormalTok{vv }\OtherTok{\textless{}{-}} \FunctionTok{variogram}\NormalTok{(zinc}\SpecialCharTok{\textasciitilde{}}\DecValTok{1}\NormalTok{, base\_sin\_outliers, }\AttributeTok{cutoff =} \DecValTok{3000}\NormalTok{, }\AttributeTok{width =} \DecValTok{200}\NormalTok{, }\AttributeTok{map=}\NormalTok{T)}
\FunctionTok{plot}\NormalTok{(vv)}
\end{Highlighting}
\end{Shaded}

\includegraphics{trabajo_final_files/figure-latex/unnamed-chunk-24-1.pdf}

Parece entonces un proceso Anisotrópico, con dirección por continuidad,
alcanzando otros valores si corremos la dirección perpendicular sentido
de la mancha.

\begin{Shaded}
\begin{Highlighting}[]
\FunctionTok{par}\NormalTok{(}\AttributeTok{mfrow=}\FunctionTok{c}\NormalTok{(}\DecValTok{1}\NormalTok{,}\DecValTok{2}\NormalTok{))}
\FunctionTok{plot}\NormalTok{(bin\_clasico,}\AttributeTok{col =} \StringTok{\textquotesingle{}cadetblue\textquotesingle{}}\NormalTok{,  }\AttributeTok{main =} \StringTok{"Variograma Empírico}\SpecialCharTok{\textbackslash{}n}\StringTok{{-} classical estimator"}\NormalTok{)}
\FunctionTok{plot}\NormalTok{(bin\_CH,}\AttributeTok{col =} \StringTok{\textquotesingle{}cadetblue\textquotesingle{}}\NormalTok{,  }\AttributeTok{main =} \StringTok{"Variograma Empírico}\SpecialCharTok{\textbackslash{}n}\StringTok{ {-} modulus estimator"}\NormalTok{)}
\end{Highlighting}
\end{Shaded}

\includegraphics{trabajo_final_files/figure-latex/unnamed-chunk-25-1.pdf}

\begin{Shaded}
\begin{Highlighting}[]
\FunctionTok{par}\NormalTok{(}\AttributeTok{mfrow =} \FunctionTok{c}\NormalTok{(}\DecValTok{1}\NormalTok{,}\DecValTok{1}\NormalTok{))}
\end{Highlighting}
\end{Shaded}

asdsaasdads

\begin{Shaded}
\begin{Highlighting}[]
\FunctionTok{par}\NormalTok{(}\AttributeTok{mfrow =} \FunctionTok{c}\NormalTok{(}\DecValTok{1}\NormalTok{,}\DecValTok{2}\NormalTok{))}
\FunctionTok{plot}\NormalTok{(bin1, }\AttributeTok{bin.cloud =}\NormalTok{ T, }\AttributeTok{main =} \StringTok{"classical estimator"}\NormalTok{)}
\FunctionTok{plot}\NormalTok{(bin2, }\AttributeTok{bin.cloud =}\NormalTok{ T, }\AttributeTok{main =} \StringTok{"modulus estimator"}\NormalTok{)}
\end{Highlighting}
\end{Shaded}

\includegraphics{trabajo_final_files/figure-latex/unnamed-chunk-27-1.pdf}

\begin{Shaded}
\begin{Highlighting}[]
\FunctionTok{par}\NormalTok{(}\AttributeTok{mfrow =} \FunctionTok{c}\NormalTok{(}\DecValTok{1}\NormalTok{,}\DecValTok{2}\NormalTok{))}
\end{Highlighting}
\end{Shaded}

Como podemos observar a través de estos gráficos podríamos observar un
ligero comportamiento \textbf{anisotrópico} en la variable, dado que
para algunos puntos el comportamiento del zinc presenta algunas
variaciones en ciertas direcciones, pero las mismas son muy pequeñas.

En este caso queremos entender si estamos frente a una variable
\textbf{anisotrópica} o \textbf{isotrópica}, es decir queremos terminar
de confirmar si el comportamiento estadístico varía según la dirección
en la que se mida la distancia, es por eso que calculamos el variograma
variando las direcciones, de esta manera queremos entender la
variabilidad espacial del zinc en diferentes direcciones. Para poder
calcularlo vamos a tomar un ángulo y distancia determinados

\begin{Shaded}
\begin{Highlighting}[]
\NormalTok{vario}\FloatTok{.2} \OtherTok{\textless{}{-}} \FunctionTok{variog}\NormalTok{(meuse, }\AttributeTok{coords =} \FunctionTok{coordinates}\NormalTok{(meuse), }\AttributeTok{data =}\NormalTok{ meuse}\SpecialCharTok{$}\NormalTok{zinc, }\AttributeTok{uvec =} \FunctionTok{seq}\NormalTok{(}\DecValTok{0}\NormalTok{, }\DecValTok{1000}\NormalTok{, }\AttributeTok{by =} \DecValTok{50}\NormalTok{), }\AttributeTok{dir=}\DecValTok{0}\NormalTok{)}
\end{Highlighting}
\end{Shaded}

\begin{verbatim}
## variog: computing variogram for direction = 0 degrees (0 radians)
##         tolerance angle = 22.5 degrees (0.393 radians)
\end{verbatim}

\begin{Shaded}
\begin{Highlighting}[]
\NormalTok{vario}\FloatTok{.3} \OtherTok{\textless{}{-}} \FunctionTok{variog}\NormalTok{(meuse, }\AttributeTok{coords =} \FunctionTok{coordinates}\NormalTok{(meuse), }\AttributeTok{data =}\NormalTok{ meuse}\SpecialCharTok{$}\NormalTok{zinc, }\AttributeTok{uvec =} \FunctionTok{seq}\NormalTok{(}\DecValTok{0}\NormalTok{, }\DecValTok{1000}\NormalTok{, }\AttributeTok{by =} \DecValTok{50}\NormalTok{), }\AttributeTok{dir=}\NormalTok{pi}\SpecialCharTok{/}\DecValTok{2}\NormalTok{)}
\end{Highlighting}
\end{Shaded}

\begin{verbatim}
## variog: computing variogram for direction = 90 degrees (1.571 radians)
##         tolerance angle = 22.5 degrees (0.393 radians)
\end{verbatim}

\begin{Shaded}
\begin{Highlighting}[]
\NormalTok{vario}\FloatTok{.4} \OtherTok{\textless{}{-}} \FunctionTok{variog}\NormalTok{(meuse, }\AttributeTok{coords =} \FunctionTok{coordinates}\NormalTok{(meuse), }\AttributeTok{data =}\NormalTok{ meuse}\SpecialCharTok{$}\NormalTok{zinc, }\AttributeTok{uvec =} \FunctionTok{seq}\NormalTok{(}\DecValTok{0}\NormalTok{, }\DecValTok{1000}\NormalTok{, }\AttributeTok{by =} \DecValTok{50}\NormalTok{), }\AttributeTok{dir=}\NormalTok{pi}\SpecialCharTok{/}\DecValTok{4}\NormalTok{)}
\end{Highlighting}
\end{Shaded}

\begin{verbatim}
## variog: computing variogram for direction = 45 degrees (0.785 radians)
##         tolerance angle = 22.5 degrees (0.393 radians)
\end{verbatim}

\begin{Shaded}
\begin{Highlighting}[]
\FunctionTok{par}\NormalTok{(}\AttributeTok{mfrow =} \FunctionTok{c}\NormalTok{(}\DecValTok{1}\NormalTok{,}\DecValTok{1}\NormalTok{))}
\FunctionTok{plot}\NormalTok{(bin\_clasico, }\AttributeTok{type=}\StringTok{"l"}\NormalTok{)}
\FunctionTok{lines}\NormalTok{(vario}\FloatTok{.2}\NormalTok{, }\AttributeTok{lty =} \DecValTok{2}\NormalTok{, }\AttributeTok{col =} \DecValTok{2}\NormalTok{)}
\FunctionTok{lines}\NormalTok{(vario}\FloatTok{.3}\NormalTok{, }\AttributeTok{lty =} \DecValTok{3}\NormalTok{, }\AttributeTok{col =} \DecValTok{3}\NormalTok{)}
\FunctionTok{lines}\NormalTok{(vario}\FloatTok{.4}\NormalTok{, }\AttributeTok{lty =} \DecValTok{4}\NormalTok{, }\AttributeTok{col =} \DecValTok{4}\NormalTok{)}
\FunctionTok{legend}\NormalTok{(}\StringTok{"bottomright"}\NormalTok{, }\FunctionTok{c}\NormalTok{(}\StringTok{"omnidireccional"}\NormalTok{, }\StringTok{"0°"}\NormalTok{, }\StringTok{"90°"}\NormalTok{, }\StringTok{"45°"}\NormalTok{), }\AttributeTok{col=}\FunctionTok{c}\NormalTok{(}\DecValTok{1}\NormalTok{,}\DecValTok{2}\NormalTok{,}\DecValTok{3}\NormalTok{,}\DecValTok{4}\NormalTok{), }\AttributeTok{lty=}\FunctionTok{c}\NormalTok{(}\DecValTok{1}\NormalTok{,}\DecValTok{2}\NormalTok{,}\DecValTok{3}\NormalTok{,}\DecValTok{4}\NormalTok{))}
\end{Highlighting}
\end{Shaded}

\includegraphics{trabajo_final_files/figure-latex/unnamed-chunk-29-1.pdf}

Podemos observar que para diferentes distancias y ángulos la variable en
cuestión presenta ciertas variaciones, por lo que estaríamos en
presencia de \textbf{anisotropía}.

Luego de este análisis pasamos a estudiar los residuos de ambos
variogramas para entender si existe un margen para analizar la
dependencia espacial.

\begin{Shaded}
\begin{Highlighting}[]
\NormalTok{res1.v }\OtherTok{=} \FunctionTok{variog}\NormalTok{(meuse, }\AttributeTok{coords =} \FunctionTok{coordinates}\NormalTok{(meuse), }\AttributeTok{data =}\NormalTok{ meuse}\SpecialCharTok{$}\NormalTok{zinc, }\AttributeTok{uvec =} \FunctionTok{seq}\NormalTok{(}\DecValTok{0}\NormalTok{, }\DecValTok{1000}\NormalTok{, }\AttributeTok{by =} \DecValTok{50}\NormalTok{))}
\end{Highlighting}
\end{Shaded}

\begin{verbatim}
## variog: computing omnidirectional variogram
\end{verbatim}

\begin{Shaded}
\begin{Highlighting}[]
\FunctionTok{set.seed}\NormalTok{(}\DecValTok{123}\NormalTok{)}
\NormalTok{s1 }\OtherTok{=} \FunctionTok{variog.mc.env}\NormalTok{(meuse, }\AttributeTok{coords =} \FunctionTok{coordinates}\NormalTok{(meuse), }\AttributeTok{data =}\NormalTok{ meuse}\SpecialCharTok{$}\NormalTok{zinc, }\AttributeTok{obj =}\NormalTok{ res1.v)}
\end{Highlighting}
\end{Shaded}

\begin{verbatim}
## variog.env: generating 99 simulations by permutating data values
## variog.env: computing the empirical variogram for the 99 simulations
## variog.env: computing the envelops
\end{verbatim}

\begin{Shaded}
\begin{Highlighting}[]
\FunctionTok{plot}\NormalTok{(res1.v, }\AttributeTok{env =}\NormalTok{ s1)}
\end{Highlighting}
\end{Shaded}

\includegraphics{trabajo_final_files/figure-latex/unnamed-chunk-31-1.pdf}

Podemos observar, usando una simulación simple de Monte Carlo, que
existe el margen para estudiar la dependencia espacial, sobre todo para
distancias cortas.

Con estos resultados procedemos a ajustar el variograma teórico, para
ello consideraremos los siguientes parámetros:

\begin{Shaded}
\begin{Highlighting}[]
\NormalTok{res1.v.ef }\OtherTok{=} \FunctionTok{eyefit}\NormalTok{(res1.v)}
\end{Highlighting}
\end{Shaded}

\includegraphics{trabajo_final_files/figure-latex/unnamed-chunk-32-1.pdf}

\begin{Shaded}
\begin{Highlighting}[]
\NormalTok{sigmasq }\OtherTok{=} \FloatTok{143151.79}
\NormalTok{phi }\OtherTok{=} \FloatTok{333.33}
\NormalTok{nugget }\OtherTok{=} \FloatTok{17893.97}
\NormalTok{modelo }\OtherTok{=} \StringTok{"exp"}
\end{Highlighting}
\end{Shaded}

\begin{Shaded}
\begin{Highlighting}[]
\NormalTok{v }\OtherTok{\textless{}{-}} \FunctionTok{variogram}\NormalTok{(zinc}\SpecialCharTok{\textasciitilde{}}\DecValTok{1}\NormalTok{, meuse)}
\NormalTok{vt\_exp }\OtherTok{=} \FunctionTok{fit.variogram}\NormalTok{(v, }\FunctionTok{vgm}\NormalTok{(}\DecValTok{190000}\NormalTok{, }\StringTok{"Exp"}\NormalTok{, }\DecValTok{1400}\NormalTok{, }\DecValTok{30000}\NormalTok{))}
\NormalTok{rango\_practico }\OtherTok{=} \FloatTok{381.7098} \SpecialCharTok{*} \DecValTok{3}
\FunctionTok{plot}\NormalTok{(v , vt\_exp, }\AttributeTok{main =} \StringTok{\textquotesingle{}Variograma Teórico Exponencial sin tendencia\textquotesingle{}}\NormalTok{)}
\end{Highlighting}
\end{Shaded}

\includegraphics{trabajo_final_files/figure-latex/unnamed-chunk-33-1.pdf}

\begin{Shaded}
\begin{Highlighting}[]
\NormalTok{vt\_sph }\OtherTok{=} \FunctionTok{fit.variogram}\NormalTok{(v, }\FunctionTok{vgm}\NormalTok{(}\DecValTok{190000}\NormalTok{, }\StringTok{"Sph"}\NormalTok{, }\DecValTok{1400}\NormalTok{, }\DecValTok{30000}\NormalTok{))}
\FunctionTok{plot}\NormalTok{(v , vt\_sph, }\AttributeTok{main =} \StringTok{\textquotesingle{}Variograma Teórico Esferíco sin tendencia\textquotesingle{}}\NormalTok{)}
\end{Highlighting}
\end{Shaded}

\includegraphics{trabajo_final_files/figure-latex/unnamed-chunk-34-1.pdf}

\begin{Shaded}
\begin{Highlighting}[]
\NormalTok{vten }\OtherTok{\textless{}{-}} \FunctionTok{variogram}\NormalTok{(zinc}\SpecialCharTok{\textasciitilde{}}\NormalTok{x}\SpecialCharTok{+}\NormalTok{y, meuse)}
\end{Highlighting}
\end{Shaded}

\begin{Shaded}
\begin{Highlighting}[]
\NormalTok{vtent\_exp }\OtherTok{=} \FunctionTok{fit.variogram}\NormalTok{(vten, }\FunctionTok{vgm}\NormalTok{(}\DecValTok{190000}\NormalTok{, }\StringTok{"Exp"}\NormalTok{, }\DecValTok{1400}\NormalTok{, }\DecValTok{30000}\NormalTok{))}
\FunctionTok{plot}\NormalTok{(vten , vtent\_exp, }\AttributeTok{main =} \StringTok{\textquotesingle{}Variograma Teórico Exponencial con tendencia\textquotesingle{}}\NormalTok{)}
\end{Highlighting}
\end{Shaded}

\includegraphics{trabajo_final_files/figure-latex/unnamed-chunk-36-1.pdf}

\begin{Shaded}
\begin{Highlighting}[]
\NormalTok{vtent\_sph }\OtherTok{=} \FunctionTok{fit.variogram}\NormalTok{(vten, }\FunctionTok{vgm}\NormalTok{(}\DecValTok{190000}\NormalTok{, }\StringTok{"Sph"}\NormalTok{, }\DecValTok{1400}\NormalTok{, }\DecValTok{30000}\NormalTok{))}
\FunctionTok{plot}\NormalTok{(vten , vtent\_sph,}\AttributeTok{main =} \StringTok{\textquotesingle{}Variograma Teórico Esferíco con tendencia\textquotesingle{}}\NormalTok{)}
\end{Highlighting}
\end{Shaded}

\includegraphics{trabajo_final_files/figure-latex/unnamed-chunk-37-1.pdf}

Como podemos observar tanto para el caso de los modelos con tendencia
como sin tendencia el exponencial es el que mejor se ajusta a nuestro
variograma. Ahora bien, la pregunta es: ¿el modelo correcto es con o sin
tendencia?, no podemos mezclar los errores cuadráticos entre modelos con
y sin tendencia.Para ello, hay que analizar los plots, pero como vimos
antes, no hay patrones claros. Además, al haber hecho variogramas con y
sin tendencia y obtener resultados tan parecidos, se refuerza nuestra
hipótesis de que no tienen tendencia.

\begin{Shaded}
\begin{Highlighting}[]
\NormalTok{modelo}\OtherTok{\textless{}{-}} \FunctionTok{c}\NormalTok{(}\StringTok{\textquotesingle{}Modelo Exponencial con tendencia\textquotesingle{}}\NormalTok{,}\StringTok{\textquotesingle{}Modelo Esferíco con tendencia\textquotesingle{}}\NormalTok{,}\StringTok{\textquotesingle{}Modelo Exponencial sin tendencia\textquotesingle{}}\NormalTok{,}\StringTok{\textquotesingle{}Modelo Esferíco sin tendencia\textquotesingle{}}\NormalTok{)}
\NormalTok{error }\OtherTok{\textless{}{-}} \FunctionTok{c}\NormalTok{(}\FunctionTok{attr}\NormalTok{(vtent\_exp, }\StringTok{\textquotesingle{}SSErr\textquotesingle{}}\NormalTok{),}\FunctionTok{attr}\NormalTok{(vtent\_sph, }\StringTok{\textquotesingle{}SSErr\textquotesingle{}}\NormalTok{),}\FunctionTok{attr}\NormalTok{(vt\_exp, }\StringTok{\textquotesingle{}SSErr\textquotesingle{}}\NormalTok{),}\FunctionTok{attr}\NormalTok{(vt\_sph, }\StringTok{\textquotesingle{}SSErr\textquotesingle{}}\NormalTok{))}
\NormalTok{df\_error }\OtherTok{\textless{}{-}}\FunctionTok{data.frame}\NormalTok{(modelo,error)}
\NormalTok{df\_error}
\end{Highlighting}
\end{Shaded}

\begin{verbatim}
##                             modelo   error
## 1 Modelo Exponencial con tendencia 1975144
## 2    Modelo Esferíco con tendencia 2668378
## 3 Modelo Exponencial sin tendencia 1791466
## 4    Modelo Esferíco sin tendencia 2223258
\end{verbatim}

Al probar con otros modelos, el exponencial sigue siendo el mejor ajuste
obtenido

\begin{Shaded}
\begin{Highlighting}[]
\NormalTok{vExp }\OtherTok{\textless{}{-}} \FunctionTok{fit.variogram}\NormalTok{(vten, }\FunctionTok{vgm}\NormalTok{(}\AttributeTok{model =} \StringTok{"Exp"}\NormalTok{),}\AttributeTok{fit.method =} \DecValTok{2}\NormalTok{)}
\NormalTok{vSph }\OtherTok{\textless{}{-}} \FunctionTok{fit.variogram}\NormalTok{(vten, }\FunctionTok{vgm}\NormalTok{(}\AttributeTok{model =} \StringTok{"Sph"}\NormalTok{),}\AttributeTok{fit.method =} \DecValTok{2}\NormalTok{)}
\NormalTok{vMat }\OtherTok{\textless{}{-}} \FunctionTok{fit.variogram}\NormalTok{(vten, }\FunctionTok{vgm}\NormalTok{(}\AttributeTok{model =} \StringTok{"Mat"}\NormalTok{, }\AttributeTok{nugget =} \DecValTok{1}\NormalTok{,}\AttributeTok{kappa =} \FloatTok{0.5}\NormalTok{),}\AttributeTok{fit.method =} \DecValTok{2}\NormalTok{)}
\NormalTok{vBes }\OtherTok{\textless{}{-}} \FunctionTok{fit.variogram}\NormalTok{(vten,}\FunctionTok{vgm}\NormalTok{(}\StringTok{"Bes"}\NormalTok{),}\AttributeTok{fit.method =} \DecValTok{2}\NormalTok{)}
\NormalTok{vSte }\OtherTok{\textless{}{-}} \FunctionTok{fit.variogram}\NormalTok{(vten,}\FunctionTok{vgm}\NormalTok{(}\StringTok{"Ste"}\NormalTok{),}\AttributeTok{fit.method =} \DecValTok{2}\NormalTok{)}


\NormalTok{vExpLine}\OtherTok{=}\FunctionTok{variogramLine}\NormalTok{(vExp,}\DecValTok{500}\NormalTok{)}
\NormalTok{vSphLine}\OtherTok{=}\FunctionTok{variogramLine}\NormalTok{(vSph,}\DecValTok{500}\NormalTok{)}
\NormalTok{vMatLine}\OtherTok{=}\FunctionTok{variogramLine}\NormalTok{(vMat,}\DecValTok{500}\NormalTok{)}
\NormalTok{vSteLine}\OtherTok{=}\FunctionTok{variogramLine}\NormalTok{(vSte,}\DecValTok{500}\NormalTok{)}
\NormalTok{vBesLine}\OtherTok{=}\FunctionTok{variogramLine}\NormalTok{(vBes,}\DecValTok{500}\NormalTok{)}

\FunctionTok{ggplot}\NormalTok{(}\AttributeTok{mapping =} \FunctionTok{aes}\NormalTok{(dist,gamma))}\SpecialCharTok{+}
  \FunctionTok{geom\_point}\NormalTok{(}\AttributeTok{data =}\NormalTok{ vten)}\SpecialCharTok{+}
  \FunctionTok{geom\_line}\NormalTok{(}\AttributeTok{data =}\NormalTok{ vExpLine,}\FunctionTok{aes}\NormalTok{(}\AttributeTok{color=}\StringTok{"Exponencial"}\NormalTok{))}\SpecialCharTok{+}
  \FunctionTok{geom\_line}\NormalTok{(}\AttributeTok{data =}\NormalTok{ vSphLine,}\FunctionTok{aes}\NormalTok{(}\AttributeTok{color=}\StringTok{"Esferico"}\NormalTok{))}\SpecialCharTok{+}
  \FunctionTok{geom\_line}\NormalTok{(}\AttributeTok{data =}\NormalTok{ vMatLine,}\FunctionTok{aes}\NormalTok{(}\AttributeTok{color=}\StringTok{"Matern"}\NormalTok{))}\SpecialCharTok{+}
  \FunctionTok{geom\_line}\NormalTok{(}\AttributeTok{data =}\NormalTok{ vSteLine,}\FunctionTok{aes}\NormalTok{(}\AttributeTok{color=}\StringTok{"Stein\textquotesingle{}s"}\NormalTok{))}\SpecialCharTok{+}
  \FunctionTok{geom\_line}\NormalTok{(}\AttributeTok{data =}\NormalTok{ vBesLine,}\FunctionTok{aes}\NormalTok{(}\AttributeTok{color=}\StringTok{"Bessel"}\NormalTok{))}\SpecialCharTok{+}
  \FunctionTok{scale\_color\_discrete}\NormalTok{(}\StringTok{"Modelo"}\NormalTok{)}\SpecialCharTok{+}
  \FunctionTok{theme\_classic}\NormalTok{()}
\end{Highlighting}
\end{Shaded}

\includegraphics{trabajo_final_files/figure-latex/unnamed-chunk-39-1.pdf}

Analizando nuevamente la \textbf{isotropía}, consideramos el
\emph{cutoff}, el cual es la máxima distancia que tiene sentido
considerar entre los puntos, en este caso es 4,200 de distancia.

\begin{Shaded}
\begin{Highlighting}[]
\NormalTok{vv }\OtherTok{\textless{}{-}} \FunctionTok{variogram}\NormalTok{(zinc}\SpecialCharTok{\textasciitilde{}}\DecValTok{1}\NormalTok{, meuse, }\AttributeTok{cutoff =} \DecValTok{4200}\NormalTok{, }\AttributeTok{width =} \DecValTok{50}\NormalTok{, }\AttributeTok{map=}\NormalTok{T)}
\FunctionTok{plot}\NormalTok{(vv)}
\end{Highlighting}
\end{Shaded}

\includegraphics{trabajo_final_files/figure-latex/unnamed-chunk-40-1.pdf}

\begin{Shaded}
\begin{Highlighting}[]
\NormalTok{vv }\OtherTok{\textless{}{-}} \FunctionTok{variogram}\NormalTok{(zinc}\SpecialCharTok{\textasciitilde{}}\DecValTok{1}\NormalTok{, meuse, }\AttributeTok{cutoff =} \DecValTok{4200}\NormalTok{, }\AttributeTok{width =} \DecValTok{500}\NormalTok{)}
\FunctionTok{plot}\NormalTok{(vv)}
\end{Highlighting}
\end{Shaded}

\includegraphics{trabajo_final_files/figure-latex/unnamed-chunk-41-1.pdf}

Como mencionamos anteriormente estamos en presencia de un proceso
\textbf{anisotrópico}, con dirección por continuidad, alcanzando otros
valores si corremos la dirección perpendicular al sentido de la mancha.

\end{document}
